\section{Introduzione}
Negli ultimi anni la computer vision è diventata un ambito di ricerca molto importante sia nel mondo accademico, sia per le sue applicazioni nel mondo reale. I due sotto-problemi principali nei quali essa si suddivide sono la detection ed il tracking.\\
Il primo problema ha come compito quello di insegnare ad una macchina ad interpretare una singola immagine mentre il secondo problema estende lo stesso compito ma nell'ambito dei video, ovvero una sequenza di immagini, correlate tra loro, dette frames. La detection è già ampiamente utilizzata in ambito commerciale e la si può per esempio trovare nei sistemi di riconoscimento facciale o di riconoscimento e lettura di testi scritti a mano. Altre sue applicazioni riguardano diagnosi mediche, controllo di prodotti industruali, analisi del territorio etc.\\ 
Il tracking è invece un argomento di ricerca più recente rispetto alla detection e trova applicazione nei sistemi di video-sorveglianza, veicoli a guida autonoma, riconoscimento di azioni etc. tuttavia alcuni di questi sistemi come per esempio i veicoli a guida autonoma non sono ancora maturi e sono tuttora oggetto di sperimentazioni.
\subsection{Scopo del progetto}
Lo scopo del progetto di stage è quello di progettare e realizzare un sistema di riconoscimento e tracciamento di specifici elementi all' interno di un video ad alta risoluzione.\\
Questo progetto comporta sfide e complessità aggiuntive rispetto all' analisi degli elementi presenti in una singola immagine sia per il fatto che un video è composto da una sequenza di frames anziché da una singola immagine, sia per il fatto che i frames trattati sono in alta definizione e quindi elaborare l'intero frame con una sola detection comporterebbe una perdita di qualità significativa.\\
Riassumendo, i due problemi principali che sono stati affrontati, in ordine sequenziale, sono i seguenti:
\begin{itemize}
\item Riconoscimento di specifici elementi in immagini con frammentazione;
\item Tracciamento di specifici elementi in un video;
\end{itemize}
Ognuno dei sotto-problemi viene prima analizzato a fondo, in seguito ne viene discussa una sua possibile soluzione ed infine viene mostrato come essa è stata realizzata ai fini di ottenere un prodotto il più performante possibile.
Alla fine, il prodotto finale viene realizzato come combinazione dei due sotto-prodotti.

\subsection{Note esplicative}
Allo scopo di evitare ambiguità a lettori esterni, si specifica che all'interno del documento verranno inseriti dei termini con un carattere 'G' come pedice, questo significa che il significato inteso in quella situazione è stato inserito nel Glossario

\subsection{Struttura del documento}
Il documento è organizzato nel seguente modo. Nel capitolo 2 viene fornita una panoramica riguardante l'azienda presso la quale è stata svolta l'attività di stage. Il capitolo 3 analizza i problemi da affrontare e le varie tecniche utilizzate nello stato dell'arte per provare a risolverli. Il capitolo 4 esamina approfonditamente gli algoritmi sviluppati per far fronte ai problemi proposti mentre il capitolo 5 spiega quali tecnologie e strumenti sono stati impiegati per la loro realizzazione. Il capitolo 6 riguarda l'implementazione degli algoritmi ed il 7 ne riporta i risultati ottenuti. Il capitolo 8 riguarda le conclusioni ed una valutazione retrospettiva dell'esperienza di stage. Infine sono presenti il Glossario e la Bibliografia.

\subsection{Struttura del lavoro}
Lo stage ha avuto una durata di circa 320 ore produttive, di queste, almeno 40 ore sono state utilizzate per lo studio delle tecnologie utilizzate, le quali, includono il linguaggio di programmazione Python e alcune delle sue librerie, tra cui Numpy, OpenCV e Tensorflow. Circa 60 ore sono state impiegate per lo studio delle basi di machine learning e computer vision e per lo studio delle soluzioni più comuni ai problemi da affrontare. Altre 40 ore sono state impiegate per ideare e progettare delle soluzioni efficaci per i risolvere i problemi descritti sulla base di soluzioni già esistenti. Lo sviluppo dei prodotti realizzati ha richiesto circa 140 ore di lavoro, compresi i relativi test. Le restanti 40 ore sono state dedicate alla raccolta ed analisi dei risultati ed alla stesura della documentazione. L'elaborazione dei risultati ha richiesto in totale circa una settimana di tempo ed è stata eseguita su una macchina appositamente dedicata e sempre funzionante in modo da poter svolgere altre attività in parallelo.
