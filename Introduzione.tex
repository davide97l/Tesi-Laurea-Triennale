\section{Introduzione}
La computer vision è un ambito dell'intelligenza artificiale il cui scopo è quello di insegnare alle macchine non solo a vedere un' immagine, ma anche a riconoscere gli elementi che la compongono in modo da poter interpretare il suo contenuto come farebbe il cervello di un qualsiasi essere umando. Nonostante le attuali tecniche di deep learning rendano possibile questo compito, è comunque necessaria una grande quantità di immagini e di tempo per poter allenare una rete neurale a sufficienza in modo da riuscire a riconoscere correttamente degli oggetti in un' immagine non incontrata durante il processo di allenamento.\\

Lo scopo del progetto di stage è stato quello di progettare e realizzare un sistema di riconoscimento e tracciamento di specifici elementi all' interno di un video ad alta risoluzione. Questo progetto ha comportato sfide e complessità aggiuntive rispetto all' analisi degli elementi presenti in una singola foto sia per il fatto che un video è composto da una sequenza di frames anzichè da una singola immagine, sia per il fatto che i frames trattati erano in formato Ultra High Definition (4K) e quindi elaborare l'intero frame in una sola volta sarebbe stato troppo oneroso dal punto di vista computazionale.\\

Riassumendo i tre problemi principali che sono stati affrontati, in ordine sequenziale, sono i seguenti:
\begin{itemize}
\item Frammentazione dell'immagine;
\item Tracciamento degli elementi;
\item Stabilizzazione delle labels degli elementi tracciati.
\end{itemize}
Ognuno dei sotto-problemi è stato analizzato a fondo, ne è stata trovata un' adeguata soluzione ed è stata applicata ai fini di realizzare un prodotto il più performante possibile.

\subsection{Note esplicative}
Allo scopo di evitare ambiguità a lettori esterni, si specifica che all'interno del documento verranno inseriti dei termini con un carattere 'G' come pedice, questo significa che il significato inteso in quella situazione è stato inserito nel Glossario

\subsection{Struttura del documento}
Il documento è organizzato nel seguente modo. Nel capitolo 2 viene fornita una panoramica riguardante l'azienda presso la quale è stata svolta l'attività di stage. Il capitolo 3 analizza i problemi da affrontare e le varie tecniche utilizzate nello stato dell'arte per provare a risolverli. Il capitolo 4 esamina approfonditamente gli algoritmi sviluppati per far fronte ai problemi proposti mentre il capitolo 5 spiega quali tecnologie e strumenti sono stati impiegati per la loro realizzazione. Il capitolo 6 riguarda l'implementazione degli algoritmi ed il 7 ne riporta i risultati ottenuti. Infine sono presenti il Glossario e la Bibliografia.
