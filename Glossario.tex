\section{Glossario}

\paragraph{B}
\subparagraph{Box}
E' un rettangolo che identifica il perimetro entro quale l'oggetto riconosciuto si trova.

\paragraph{C}
\subparagraph{Categoria}
L'obiettivo della classificazione è quello di assegnare all'oggetto individuato la sua categoria di appartenenza.

\paragraph{L}
\subparagraph{Label}
Etichetta che viene data ad ogni elemento riconosciuto e ne indica la categoria di appartenenza, una detection box ed uno score.

\paragraph{O}
\subparagraph{Overlap} Si ha un overlap quando due regioni hanno un'area in comune. L'Overlap è anche un attributo da tenere conto in fase di frammentazione di un'immagine ed indica di quanto spazio una regione deve espandersi oltre alla sua area originale in modo da ottenere una sovrapposizione con le aree adiacenti.

\paragraph{R}
\subparagraph{Raggruppamento}
Insieme di labels correlate tra loro tale che da una qualsiasi label del gruppo sia possibile raggiungere qualsiasi altra label dello stesso insieme passando solo per label vicine \footnote{Due label sono considerate vicine se sono intersecate tra loro ed intersecano anche almeno una zona di sovrapposizione}.
\subparagraph{Regione} Una sezione di un'immagine con un'area di dimensione minore dell'area dell'immagine originale.

\paragraph{S}
\subparagraph{Score} La probabilità che la categoria associata all'elemento riconosciuto sia quella corretta.
\subparagraph{Soglia} E' il valore che lo score di una label deve eguagliare o superare per essere considerata affidabile.

\paragraph{T}
\subparagraph{Tolleranza} La distanza in pixels per la quale una label può discostare da una zona di interesse per cui sia ancora considerata essere intersecata con la zona di interesse.
