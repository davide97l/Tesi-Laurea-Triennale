\section{Conclusioni}

\subsection{Esperienze acquisite}
Il lavoro di stage è stato molto ampio e formativo, i corsi seguiti durante la laurea triennale sono stati sicuramente una buona base di partenza per lo svolgimento del progetto ma non sufficienti per poter comprendere i concetti messi in pratica durante il lavoro. Il corso di Ingegneria del Software mi ha permesso di organizzare meglio il mio lavoro suddividendo il ciclo di vita del progetto in fasi distinte e migliorando la qualità del software sviluppato tramite l'applicazione di best practices e pattern architetturali.\\
La maggior parte della fase di apprendimento è stata svolta da autodidatta studiando da diversi papers scientifici riguardanti gli argomenti trattati, in questo modo ho potuto approfondire le tecniche attualmente utilizzate nell'ambito della computer vision sia allo stato dell'arte che altre già più consolidate.\\
Questa esperienza è stata molto positiva in quanto mi ha permesso di approfondire argomenti che non sono stati trattati durante il corso della laurea triennale ma che saranno molto utili in futuro in quanto questo tipo di tecnologie sono sempre più presenti negli attuali sistemi informatici trovando una loro applicazione in molti ambiti distinti e non solo all'interno dell'informatica.

\subsection{Difficoltà incontrate}
Lo svolgimento del progetto non è stato privo di difficoltà. A causa di mancate conoscenze pregresse, la progettazione dell'algoritmo di frammentazione e del sistema di tracking non hanno dato fin da subito risultati positivi e solo dopo diversi tentativi e ricerche è stata trovata una soluzione accettabile per entrambi. Per questo motivo alla fase di progettazione è stato dedicato più tempo del previsto ed ha costretto a ripetere più volte alcune parti della fase di sviluppo. Tuttavia, la fase di acquisizione dei risultati, anche grazie alla sua parallelizzazione, è stata più rapida del previsto permettendo di finire il lavoro entro i tempi programmati.\\
Un'altra difficoltà ha coinvolto la detection vera e propria: nella detection con frammentazione è stata utilizzata una rete neurale progettata ed allenata dall'azienda per riconoscere oggetti specifici. Nei video sui quali è stato testato il sistema di tracking erano presenti diversi tipi di categorie rispetto a quelle del dataset utilizzato per valutare la detection, non potendo quindi riutilizzare lo stesso modello. Per ovviare il problema è stato quindi utilizzato un semplice modello allenato sul dataset Coco. Di conseguenza, a causa della scarsa qualità del nuovo modello, i risultati del tracking con detections reali sono molto più scarsi rispetto a quelli con detections perfette ed un modello migliore avrebbe sicuramente permesso di ottenere risultati più apprezzabili. 