\section{Progettazione}
\subsection{Algoritmo per la frammentazione dell'immagine}
Per quanto riguarda il problema della frammentazione dei frames in 4K è stato individuato un apposito algoritmo in grado di effettuare il riconoscimento degli oggetti in nei frames presenti senza doverli ridimensionare ma utilizzando invece una tecnica conosciuta come frammentazione dell'immagine.
\subsubsection{Scomposizione del frame originale in regioni}
Il frame in 4K viene scomposto in una matrice di n x m sotto-immagini chiamate regioni\gls in modo tale che ogni regione sia efficacemente analizzabile da una Faster R-CNN. Per facilitare l'operazione di detection da parte della rete, ogni regione si sovrappone leggermente con le sue regioni adiacenti. Per definire la quantità di pixels da coinvolgere nella sovrapposizione viene definito uno stride\gls che indica quanti pixels della regione tralasciare, sia in verticale che in orizzontale, prima che cominci quella successiva, ovviamente lo stride deve essere minore della larghezza di una regione.
Una Faster R-CNN dopo aver elaborato singolarmente ogni regione darà in output una lista di labels\gls con le seguenti caratteristiche:
\begin{itemize}
\item \textbf{(min-x, min-y)}: coordinate del vertice in basso a destra del rettangolo rappresentante la label dell'oggetto riconosciuto; 
\item \textbf{(max-x, max-y)}: coordinate del vertice in alto a sinistra del rettangolo rappresentante la label dell'oggetto riconosciuto; 
\item \textbf{Classe}: è un numero naturale che indica la categoria\gls di appartenenza dell'elemento individuato;
\item \textbf{Score\gls}: rappresenta la misura di probabilità che la classificazione sia effettivamente quella corretta.
\end{itemize}
Successivamente viene aggiustata la posizione delle labels individuate in modo da traslarle nella loro posizione corretta all'interno dell'immagine originale non frammentata. Questo viene fatto aggiungendo un adeguato offset alle coordinate dei vertici dei box delle labels sulla base della loro regione di appartenenza.
\subsubsection{Rimozione degli elementi individuati più volte all'interno delle aree di sovrapposizione}
A causa della presenza delle aree di sovrapposizione dovute alla struttura delle regioni, gli elementi giacenti in queste particolari zone del frame verranno individuati tante volte quante sono le regioni che si sovrappongono in quella determinata zona. Per eliminare le copie duplicate e tenerne solo una viene utilizzato un algoritmo chiamato Average Non-Max Suppression (ANMS) che è una variante del Non-Max Suppression tipicamente utilizzato dalle reti di tipo RCNN e le sue varianti. Invece che tenere la label con lo score maggiore ed eliminare tutte le altre, come box viene usato il box che deriva dalla media dei box di tutte labels e lo score viene calcolato come la media delle loro scores.
Questo metodo è fondato sul ragionamento che non bisognerebbe buttare via delle informazioni già possedute ma piuttosto riutilizzarle per scoprire qualcosa di nuovo. Ad uno stesso elemento visualizzato dentro due sezioni differenti di un' immagine potrebbero venirgli assegnati due score diversi. Mentre NMS conserverebbe solo il valore più alto tra i due, ANMS li utilizzerebbe entrambi per ottenere un valore ancora più affidabile.
\subsubsection{Creazione di raggruppamenti di labels correlate} 
A questo punto tutti gli elementi sono stati individuati e classificati ma rimane comunque il problema che a causa della precedente scomposizione, gli oggetti situati in prossimità o all'interno delle aree di scomposizione risulterebbero individuati due o più volte. Questo numero varia in base al numero di regioni sulle quali giace l'oggetto come riportato in figX. Il secondo problema è che due elementi vicini \footnote{Due label sono considerate vicine se appartengono a due regioni confinanti e sono situate dentro o in prossimità di un'area di sovrapposizione}, anche se classificati nella stessa categoria, non è detto che necessariamente debbano rappresentare lo stesso elemento. Un esempio di questo caso lo si può notare in figX. Un caso ancora peggiore è quello mostrato in figX dove non solo l'elemento è situato su più regioni differenti ma sussiste anche il problema che ogni parte dell'elemento verrebbe classificata in modo diverso a causa della loro ambiguità. Infine, la figX mostra un oggetto che si distribuisce su molte regioni ed ogni sua label presenta dimensioni diverse.
La soluzione individuata consiste nel raggruppare label correlate tra loro in insiemi di labels dette raggruppamenti\gls.
Inizialmente vengono individuate tutte le labels situate in prossimità dei confini di regioni, ovvero quelle labels che intersecano le aree di sovrapposizione o che non distino più di un fissato numero di pixels, detto \gls{tolleranza}, da esse. 
Per creare i raggruppamenti di labels è stato ideato il seguente algoritmo:
\begin{enumerate}
\item Vengono tenute solo le labels vicino ad un confine di regione e sono considerate come \textit{libere};
\item Seleziona una label \textit{libera} e la fa diventare \textit{controllata};
\item Per ogni label \textit{controllata} controlla se ci sono altre labels che la intersecano o che siano distanti entro la tolleranza fissata e che rispettino una \textbf{condizione di verità};
\item Le labels così trovate diventano a loro volta \textit{controllate};
\item Si ripetono i punti 3 e 4 fino a che non sia più possibile trovare ulteriori labels;
\item Tutte le label \textit{controllate} vengono ora classificate come \textit{raggruppate} e viene assegnato un numero progressivo ad ogni label \textit{controllata} in modo da identificarne il gruppo di appartenenza;
\item Si ripetono i punti da 2 a 6 fino a che tutte le labels non vengano raggruppate.
\end{enumerate}
\textbf{Condizione di verità}: Per effettuare un corretto raggruppamento delle labels viene anche tenuta in considerazione la categoria a loro associata tramite la classificazione insieme allo score assegnato. Per definire il risultato della condizione è inoltre necessario stabilire una soglia\gls di probabilità per considerare un'etichetta come affidabile o meno. Di seguito vengono riportati i vari casi per decidere se la condizione è vera o falsa.
\begin{itemize}
\item \textbf{True}: Le due labels hanno la stessa categoria ed entrambe con score uguale o maggiore della soglia;
\item \textbf{True}: Le due labels hanno la stessa categoria ma almeno una delle due ha score minore della soglia;
\item \textbf{True}: Le due labels hanno categoria diversa ma almeno una delle due ha score minore della soglia;
\item \textbf{False}: Le due labels hanno categoria diversa ed entrambe con score uguale o maggiore della soglia;
\end{itemize}
E' da notare che labels intersecanti ma nella stessa regione non sono motivo di interesse in quanto l'algoritmo di Non-Maximum Suppression utilizzato dalla rete in fase di post-processing ci assicura che labels intersecanti individuino elementi diversi.
In seguito bisogna trasformare ogni raggruppamento in una nuova label che racchiuda tutte le labels che lo compongono. Per fare questo vengono esaminate le coordinate di ogni vertice di tutte le labels di un raggruppamento in modo tale da trovare quattro nuovi vertici di un rettangolo che soddisfi i requisiti sopra discussi. Il nuovo box così creato andrà a sostituire le labels del rispettivo raggruppamento e come etichetta verrà tenuta l'etichetta posseduta dalla label con score maggiore. A questo punto l'algoritmo può dirsi concluso ed è in grado di riconoscere gli elementi in un'immagine in 4K con un'accuratezza accettabile e buona velocità. Tuttavia in casi particolari come quello mostrato in figura figX l'algoritmo commetterebbe un errore in quanto individuerebbe due individui con una sola label.
\subsubsection{Raggruppamenti di labels utilizzati come region proposal}
Un ulteriore miglioramento dell'algoritmo viene ottenuto utilizzando le labels ottenute dal procedimento descritto in precedenza come nuove regioni sulle quali applicare nuovamente Faster R-CNN per identificare nuovamente gli elementi contenuti nella regione ma con maggiore precisione in quanto questa volta l'area non verrà affetta da problemi di frammentazione dando quindi la possibilità alla rete di esaminare l'oggetto per intero. La regione viene prima inizializzata rimuovendo la sua label e poi ripopolata dalle nuove labels identificate dalla rete.
Il primo problema che salta fuori è che durante questo procedimento la rete identificherà nuovamente anche quegli elementi che casualmente si trovavano dentro la regione coinvolta ma che erano già stati trovati anche in precedenza. Tuttavia questo problema viene tranquillamente risolto applicando un algoritmo di Average Non-Max Suppression, utilizzato già in precedenza, per eliminare oggetti quasi completamente sovrapposti. Il secondo problema riguarda ancora gli oggetti che stanno a cavallo tra la regione interessata e l'immagine originale, questa volta però, avendoli già individuati nella loro interezza durante la prima fase è quindi solamente necessario integrare la nuova label con quella già trovata in precedenza. 
Un caso particolare lo si ha quando la la regione in esame risulti essere così estesa da vanificare i vantaggi ottenuti dalla frammentazione. Per far fronte a questo problema basta ridimensionare l'area coinvolta fino a portarla ad avere dimensioni gestibili da una rete. In questo caso la perdita di risoluzione e quindi di dettagli non comporterebbe un grave problema in quanto gli elementi visibili solo grazie all'alta definizione sono già stati individuati nella fase precedente. Nel caso in cui dovessero venire nuovamente identificati verrebbero gestiti dall'ANMS per ottenerne una migliore approssimazione. Questa funzionalità permette di migliorare l'accuratezza quando si vogliono identificare oggetti che si estendono su due o più regioni o per migliorare la detection di gruppi di elementi molto vicini tra loro ed in prossimità di un confine. 


