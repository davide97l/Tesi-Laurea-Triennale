\section{Tecnologie}
\subsection{Python}
Il linguaggio di programmazione utilizzato per perseguire gli obiettivi del progetto è stato Python\textit{v3.7}, si tratta di un linguaggio di programmazione di alto livello il cui obiettivo è quello di facilitare la leggibilità del codice ed adattarsi a diversi paradigmi di programmazione come quello procedurale, ad oggetti e funzionale. La motivazione per la scelta dell'utilizzo di questo linguaggio, oltre alla sua semplicità, è per il suo supporto di numeri frameworks e moduli relativi al deep learning e alla computer vision tra i quali Tensorflow e OpenCV. Qualsiasi modulo aggiuntiva può essere semplicemente installata eseguendo il comando:
\begin{verbatim}
pip install nome_modulo
\end{verbatim}
Altri moduli che sono stati utilizzati comprendono Numpy e Matplotlib.
\subsection{Pycharm}
Pycharm è un IDE per programmare in Python sviluppato da JetBrains. Le sue caratteristiche più importanti includono: 
\begin{itemize}
\item Un sistema intelligente di completamento automatico del codice;
\item Analisi statica del codice eseguita a tempo di esecuzione;
\item Individuazione e risoluzione veloce degli errori tramite proposte di correzione;
\item Possibilità di lavorare in un ambiente di sviluppo virtuale dove per ogni progetto vengono installate solamente le proprie dipendenze e i propri moduli. 
\end{itemize}
\subsection{Tensorflow}
Tensorflow è un framework open-source per lo sviluppo e l'allenamento di modelli di machine learning. NON ANCORA USATO
\subsection{OpenCV}
NON ANCORA USATO
\subsection{Numpy}
Numpy è un modulo di Python che fornisce il supporto per
la gestione di matrici e array multidimensionali di grandi dimensioni. Dispone anche di una vasta collezione funzioni matematiche per lavorare ad alto livelli si di essi. Viene spesso utilizzato per operare su grandi quantità di dati in modo rapido ed efficiente.
\subsection{Matplotlib}
\subsection{Pytest}